NekoHit Project总结了上一节中已有模式的不足,并意图使用区块链和智能合约技术解决
创作者与支持者之间的信任问题。该企划由三个关键部分组成:作品完成度协议(WCA)、完成
协议代币(CAT)与社区治理代币(Nekoin)。我们的企划注重实现一个简单便捷好用的赞助平台,
并且给予用户(创作者和赞助者)充分保护自己权益的权力与能力,以此来消除前述的信任问题。
同时我们尽可能保持开放:我们并不限制创作者使用何种平台存放作品,也不限制用户以何种方式
调用我们的合约,在未来我们还将尽力实现用户自由选择使用何种代币进行质押与赞助。

\subsection{作品完成度协议}\label{subsec:wca}

作品完成度协议(WCA)是NekoHit Project的核心功能。WCA以项目为基本单位,在传统模式的基础
上赋予赞助者更多主动权,除了可以随时退款\footnote{
    退款取决于项目的进行进度,在某些情况下无法全额退款
}之外,还提供类似保险的机制,在创作者未能如期完成约定时将自动退款至赞助者的钱包,并对创作者做出惩罚。
该协议在企划中以智能合约的形式实现。

\subsubsection{概念解释}

\paragraph{项目}

指创作者在现实世界中发起的活动,例如小说、插画或其他作品的制作。项目在WCA合约中记录为一系列
必要数据,例如创作者的地址、项目的里程碑定义等数据。合约不关心创作者的项目具体是是什么,如何
完成,而只要求创作者在声明项目时提交这些必要的数据。每个项目具有一个唯一的文字标识符,以供
用户在与合约交互过程中唯一的指定某个项目进行操作。

\paragraph{里程碑}

每个项目由至少一个里程碑组成,里程碑对应现实世界中项目的关键事件。对于持续性的活动,里程碑
可以对应每个周期内的第几次更新;对于一次性的众筹项目,里程碑可以对应该项目的重要环节。NEP-17
资产的结算会根据各里程碑的状态计算得出。

每个里程碑具有预期完成时间,创作者只能在该时间点之前将里程碑标记为完成。逾期则视为未完成,
最终结算时会将该里程碑对应的赞助资产退回给支持者,并按照该里程碑的权重补偿支持者。

\paragraph{阈值里程碑}

该里程碑由创作者指定,称为一个特殊的里程碑。在语义上,该里程碑达到(标记为完成或逾期)之前,
项目的参与者可以反悔。即创作者可以选择撤销项目(所有人不会有损失),赞助者也可以选择全额退款。
该里程碑达到之后,创作者不能撤销项目,强行结束项目将会导致创作者损失大部分质押的资产,
并无法获得大部分支持者的赞助。同样的,支持者也不能进行全额退款,只能获得尚未完成的里程碑所
对应的资产,对于已完成的部分将直接发送给创作者。

\paragraph{冷却时间}

创作者需要自行调用合约以将不同的里程碑标记为完成。在这个过程中创作者必须设定一个冷却时间,
确保两次里程碑更新的最短时间间隔不小于该冷却时间。设定冷却时间的目的在于保护赞助者,使他们
能够及时发现恶意项目并退款。由于不同项目对最短更新间隔的要求不同,因此冷却时间将由创作者指定,
并在赞助者浏览项目时以明显方式展示,由赞助者决定是否相信创作者设定的冷却时间。

\paragraph{最大赞助数额}

不同于传统的赞助模式,WCA规定一个项目可以接受的总赞助金额必须存在上限,这样才能够保证创作者
违约时每个支持者仍然可以获得合理的赔偿数额。所有赞助者对项目的赞助金额不能超过该值。

\paragraph{质押比率}

该值与最大赞助数额共同决定了质押资产的总额度,该值也是违约时计算赔付的比率。

\paragraph{状态与阶段}

为了便于管理项目,WCA合约中使用状态来静态的标记一个项目所处的进度。例如退款时要求项目的
状态不能为“已完成”或“未开始”。状态仅在用户调用合约对项目进行操作时才会更新,例如创作者
质押正确数量的资产将会使项目从“未开始”变成“正在进行”。

每个状态可以拥有0个或数个阶段,阶段是调用时动态计算得到的。例如“正在进行”状态中,根据里程碑
的完成状态,会有“开启”(阈值里程碑尚未完成,可以反悔)、“活跃”(阈值里程碑已经完成,不能反悔)
和“待结束”(最后一个里程碑已经达到,但资产尚未结算)等不同阶段。不同的操作对项目的状态与阶段
有不同的要求,并且一些操作还会因为状态和/或阶段的不同而产生不同的行为。

\subsubsection{流程总览}

WCA合约的一次完整使用流程如下:

\begin{enumerate}
    \item 创作者调用 WCA 合约声明项目
    \item 创作者调用 NEP-17 合约转账(质押代币)
    \item 支持者调用同一个 NEP-17 合约转账(赞助项目)\label{item:purchase_wca}
    \item 创作者调用 WCA 合约更新项目进度(更新里程碑)\label{item:update_milestone}
    \item 创作者或赞助者调用 WCA 合约进行项目结算(结束项目)\label{item:finish_project}
\end{enumerate}

其中步骤\ref{item:update_milestone}可由创作者多次调用,以更新不同的里程碑。
赞助者可以从步骤\ref{item:purchase_wca}转账赞助的一刻开始,到步骤\ref{item:finish_project}
项目结束前随时退款,退款数额依照项目进行情况,从全额到部分不等。

\subsubsection{声明项目}

声明项目是指将给定的项目信息与标识符绑定,并存储到WCA合约的存储区中。声明式填写的信息之后
将无法修改。

\subsubsection{质押代币}

质押代币要求创作者在声明项目后,将指定数额的代币转账到WCA合约的账户中,随后才允许
赞助者购买。若尝试赞助未质押的项目,或对未质押的项目进行操作,合约将抛出异常并使当次交易无效。

质押数额的计算公式如下:$\text{质押总额度} = \text{质押比率} \times \text{最大赞助数额}$

假设项目A的最大赞助金额为1000代币,质押比率为0.5,则总共需要质押500代币。

目前WCA合约只支持CAT代币,但未来将支持更多代币。

\subsubsection{赞助项目}

创作者完成质押后,可以在其他平台宣传自己的项目。赞助者可以凭借项目标识符进行赞助。

赞助方法是将代币转账到WCA合约的账户,转账时需要将项目标识符作为转账的第四个参数(即data)。
收到转账后合约将检查项目标识符是否有效,若项目标识符不存在或项目不可购买,合约将抛出异常,
使本次交易无效,因此您的资产不会遭受任何损失。

项目可购买的条件是:创作者已进行质押,并且,项目的最后一个里程碑未完成或未逾期。

\subsubsection{更新项目}

创作者可随时调用WCA合约来更新项目的里程碑。更新项目时只可更新后续的里程碑,而不可更新已经经过的里程碑。

例:目前项目更新了第1个和第3个里程碑,此时第2个里程碑即便没有逾期,也无法被更新。

更新里程碑时作者必须附加一段文本。由于区块链存储区的写入费用较高,因此我们并不推荐将您的
作品直接存储在区块链中。这段文本可以是一个URL,也可以是一个IPFS CID,或一段指引您的赞助者
找到您成果的文字。若赞助者未能通过这段文本找到您的成果,他们可能会认为您开始恶意完成里程碑,
并考虑退款。里程碑一旦更新,将不可修改,因此请在正式发起交易前再三检查这段文本。

另请注意:所有存储在Neo N3区块链上的数据都是公开透明的,这意味着您的成果可以被
任何人(包括之前退款的赞助者)读取。我们准备使用NeoFS解决该问题。

\subsubsection{退款}

赞助者可以随时在项目结束前发起退款,退款将有合约自动处理,无需等待管理员或创作者同意。
具体退款规则如下:

若项目的阈值里程碑尚未完成,您可以随时进行全额退款,退款后您的购买记录将被删除。
作者将不会知晓您的退款,除非查阅区块链的交易记录。

若退款时已经经过了项目的阈值里程碑,您只能得到作者尚未完成的里程碑所对应的比例。
以如下项目为例:一共10个里程碑,第一个编号为1,最后一个编号为10,阈值里程碑为第3个。
申请退款时第1个、第2个和第4个里程碑已经完成,此时第3个里程碑没有完成,虽然没有逾期,
由于第4个里程碑已经完成,因此认定第3个里程碑已经经过,故判定只能部分退款。

退款比例为:10个里程碑作者完成了3个,因此30\%的赞助数额将直接发送给创作者,
余下70\%的代币将通过转账的形式发送给赞助者的账号。在最终结算时,第3个里程碑
会被判定为未完成,因而会有10\%的质押代币按比例分给所有赞助者作为违约的补偿,
但退款时赞助者不会获得这部分代币。

\subsubsection{结束项目}

结束项目将导致WCA合约清算项目的财产,并开始将代币转账给每一个相关的账户。
以如下项目为例:某项目最大赞助数额为5000,质押比例为0.2,共计10个里程碑,
创作者完成了其中7个,创作者质押了1000代币。

用户A赞助了500代币,用户B赞助了1000代币,用户C赞助了300代币。

结算时,用户A对应的质押金额为$500 \times 0.2 = 100$代币,因此用户A的
总结算金额为赞助金额500代币加上质押部分的100代币,总共600代币。创作者
违约了3个里程碑,因此其中$\frac{3}{10} \times 600 = 180$(零头向下
取整)代币转账给赞助者。类似的,用户B将得到360代币的退款,用户C将得到
108代币。

对于打给创作者的部分,按如下方式计算:质押部分加所有用户购买的数量,
即$1000 + 500 + 1000 + 300 = 2800$代币,这些代币减去转账给赞助者
的金额,余下$2800 - 180 - 360 - 108 = 2152$代币,这些代币将一次性
转账给创作者。

更新最后一个里程碑时将自动触发该阶段。

\subsection{完成协议代币}\label{subsec:cat}

完成协议代币(Completion Agreement Token,简称CAT)是NekoHit Project发行的流通代币。
同时,CAT代币也是目前WCA合约唯一接受的代币。

CAT代币的设计初衷即在于完全消除市场波动对于代币价值的影响。CAT代币应当主要用于在NekoHit
Project中支持内容创作者,因此采用固定汇率与开放铸币权的方式发行。CAT代币与美元锚定币的兑换
汇率为1CAT兑0.5美元锚定币。总发行量不固定,精度为小数点后两位,与现实中的法定货币一致。
用户通过调用指定美元锚定币向CAT代币的合约地址转账,CAT合约将根据固定汇率自动铸对应数量的代币
记录到转账来源地址的账户中,用户也可以通过调用销毁代币的合约方法,销毁一定数量的CAT代币,并将
等值美元锚定币退还给用户。出于安全考虑,铸币与销毁代币不允许合约地址调用。NekoHitDev开发组
将兑换一定数量的CAT代币用于空投。

由于目前Neo Legacy上的DEX尚未迁移到N3主网,N3上还没有适合的美元锚定币或算法稳定币,因此
目前CatToken将实现为一个发行量较低的普通Nep-17代币,并在后续出现合适币种后通过升级的方式
转换为上述实现。

\subsection{社区治理通证}\label{subsec:nekoin}

社区治理通证Nekoin(Neko和Coin的合成词)用于NekoHit Project社区成员对该企划的治理。
例如通过社区投票决定上述智能合约是否要新增或删除功能等事务。持有该通证意味着对该企划具有
治理权,该通证本身不具备任何价值,持有者可以使用该通证进行流通或与其他代币进行兑换,但
这并非该通证的设计用途。
