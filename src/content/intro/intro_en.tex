The NekoHit Project was built on top of the work completion agreement (WCA
for short).
The project aims to incorporate \textit{insurance} into the traditional sponsor
model, promote audiences' willingness to support content creators financially,
and motivate creators to comply with their promises and finish the project
on time.

NekoHit Project use a project-based sponsor model.
Instead of paying a monthly subscription fee, we allow the audience to sponsor
a project more precisely.
To gain the audience's trust, content creators need to pledge (stake) some
tokens first.
These tokens need to be transferred to our WCA smart contracts before the
sponsors can transfer the tokens to the creator's projects for sponsorship.
Our WCA contract also holds the sponsored tokens.
When declaring a project, the creator needs to provide information such as
the staking (pledging) rate, detailed milestones, and the expected completion
time.
Suppose the creator does not complete certain milestones by the scheduled time.
In that case, the WCA contract will automatically calculate the percentage of
those milestones to the project and deduct it from the pre-pledged tokens.
After the project is completed, the deducted tokens will be released to each
sponsor in proportion to the reference sponsorship amount.
Also, tokens sponsored by sponsors will be deducted by the corresponding
percentage and returned to the sponsor's wallet as a refund.
So not only does the author not receive the sponsorship tokens for his defaulted
milestone, but he also loses some of his own pledged tokens.

We use this mechanism as a safety net to protect our sponsors' rights and
incentivize creators to complete their projects in time.
Such a mechanism will also urge the creators to think carefully about the
project and arrange the timeline appropriately.
Last but not least, we hope that this will increase the audience's trust in
the content creators and reduce their concerns about financial sponsorship.
