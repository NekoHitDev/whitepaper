“创作者经济”在目前越来越受到人们的关注,读者和观众不仅有意愿去资助创作者,创作者也需要
各种渠道去获得收入来维持自己的创作。

目前关于创作者经济的模式主要分为订阅模式和众筹模式。

最典型的订阅模式就是Patreon了。该平台基于按周期订阅的模式,创作者可以自定义不同金额的计划,
通常较低金额的计划享有的回报较少,而较高金额的计划则对应享有更多回报。订阅模式要求创作者
事先已经完成了一部分作品,并且之后具有定期更新的作品或动态。一旦支持者对创作者交付的作品
或交付频率产生不满,支持者只能选择从下个订阅周期开始停止订阅,对于已经支付的订阅费用则无能
为力。

遵循众筹模式的平台,以Kickstarter为例,则更倾向于一次性的赞助方式。与Patreon类似,创作者
可以设定不同金额的计划,支持者一次性支付对应数额,即可享有该计划约定的回报。根据Kickstarter
的介绍\cite{kickstarter_about},该平台致力于将创意转换为现实。众筹项目的发起人将对作品
具有完全的控制权。这意味着Kickstarter更偏向于服务创作者,给他们尽可能多的自由以使他们能够
将优秀的创意转换为现实中的产品。这种创作者的独立性能够激发出卓越的产品,可一旦遇到恶意项目,
当赞助者发现他们上当了的时候,他们几乎做不了什么。

众筹模式将主动权全部放在了创作者身上,一旦赞助者支付了赞助钱款,之后项目能否完成,
赞助者能否得到产品或退款,则完全听天由命。而订阅模式中,虽然赞助者可以选择下个周期
起停止订阅,但对于已经订阅计划的赞助者来说,创作者就算什么也不做,或者更新质量变差,
他们能做的也只有停止订阅。

由此可见,在现有的创作者经济模式中,支持者处于相对弱势的地位,他们在面对恶意创作者时
无法有效地保护自己的权利,并且难以事先发现恶意的创作者。这种不信任可以被创作者的信誉
弥补,但对于刚刚开始创作的创作者来说,这种信任危机是他们获得支持资金的外部壁垒。