There is a growing interest in the creator economy, where audiences are
willing to fund creators, and creators also need various sources of income
to sustain their work.

Currently, two famous models are widely used on the market: subscription-based
model and crowdfunding-based model.

The typical platform using the subscription-based model is Patreon.
It is based on a monthly subscription model, where creators can customize
their plans at different prices.
Usually, the lower price means fewer returns, and higher the price, more 
returns you get.
Subscription-based model requires creator has already finish part of his work,
meanwhile having reasonable updates to response their supporters.
Once supporters not satisfied with the update, they can only stop the subscription,
and can't do much about the already paid subscription.

Kickstarter is a platform using the crowdfunding-based model.
It prefers a one-time sponsorship compared to Patreon's monthly subscription.
Just like Patreon, creator can set up different returns for different amount.
According to Kickstarter's introduction\cite{kickstarter_about}, the platform
is dedicated to transforming creative ideas into reality.
The initiator (creator) of a crowdfunding project will have complete control
over the works.
Kickstarter favors creators by giving them as much freedom as
possible to turn excellent creative ideas into real-world products.
Such independence can inspire superior products, but supporters can do
nothing when they realize they have been fooled in malicious projects.

The traditional crowdfunding model puts all the initiative on the creator.
Once the sponsor pays the sponsored money, they have no control over the
project, no matter it can be completed, or be refunded, or never updated again.
For a monthly subscription model, sponsors can stop subscribing from
next month.
However, for sponsors already subscribed, the only thing they can do is stop
the subscription if the creator doesn't do anything or delivered less-quality
works.

As a conclusion, in current sponsor models, supporters are in a weak position
where they cannot protect their right in an efficient way when dealing with
malicious creator.
This type of creator can be hard to locate and uncover, thus the existence of them
make supporters trust less when facing a freshmen and preventing fresh creators
to start their project.
